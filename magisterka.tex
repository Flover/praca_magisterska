\documentclass[brudnopis]{xmgr}
\usepackage{amsrefs}

%\defaultfontfeatures{Scale=MatchLowercase}
%\setmainfont[Numbers=OldStyle,Ligatures=TeX]{Minion Pro}
%\setsansfont[Numbers=OldStyle,Ligatures=TeX]{Myriad Pro}
% for fontspec version < 2.0
\setmainfont[Numbers=OldStyle,Mapping=tex-text]{Minion Pro}
\setsansfont[Numbers=OldStyle,Mapping=tex-text]{Myriad Pro}
%\setmonofont[Scale=0.75]{Monaco}

% Opcjonalnie identyfikator dokumentu
% drukowany tylko z włączoną opcją 'brudnopis':
\wersja   {wersja wstępna [\ymdtoday]}

\author   {Mateusz Kwiatkowski}
\nralbumu {194\,925}
\email    {emflover@gmail.com}

\kierunek{\textbf{INFORMATYKA}}

\title    {Walidacja w aplikacjach Meteor}
\date     {2014}
\miejsce  {Gdańsk}

\opiekun  {dr Włodzimierz Bzyl}

% dodatkowe polecenia

\begin{document}

\begin{abstract}
Jednym z założeń, które chcę osiągnąć  przy tworzeniu pakietu walidującego elektroniczny indeks jest
wystawianie ocen z dostępnej skali.  Kolejnym problemem który należy rozwiązać jest brak weryfikacji
dotychczasowych osiągnięć studenta przy wystawianiu ocen z egzaminu co pozwala na wystawienie oceny
pozytywnej przy braku zaliczenia zajęć z danego przedmiotu. Następnym etapem będzie obliczanie średniej
ocen na podstawie wyników jedynie z egzaminów. Podczas tworzenia aplikacji musi zostać spełniony
warunek okresu zmian w systemie aby nie dopuszczać do sytuacji w której wystąpiłaby możliwość zmian
ocen z poprzednich semestrów.
\\
Celem pracy jest stworzenie pakietu walidującego elektroniczny indeks w frameworku \textit{MeteorJS} co pozwoli
na usprawnienie funkcjonowania aplikacji.
 
\end{abstract}
%\keywords{SGML,
% dokumenty strukturalne}

% tytuł i spis treści
\maketitle
%
% wstęp
\introduction

Walidacja jest działaniem mającym na celu potwierdzenie w sposób udokumentowany i zgodny
z założeniami, że procedury, procesy, urządzenia, materiały, czynności i systemy rzeczywiście
prowadzą do zaplanowanych wyników. Wykorzystuje się ją w naukach technicznych oraz informatyce.
Przykładem wykorzystania tej techniki jest elektroniczny indeks.
\\
\\
Jeszcze do niedawna na wszystkich uczelniach stosowano klasyczne indeksy papierowe,
jednak w wyniku rozwoju technologii internetowych coraz częściej rezygnuje się z klasycznych
rozwiązań zastępując je ich elektronicznymi odpowiednikami. Elektroniczne indeksy pozbawione
walidacji pozwalają użytkownikowi na wprowadzenie irracjonalnych danych do systemu.
Przykładem takiego działania jest wystawienie studentowi oceny spoza skali czy też wystawienie
oceny końcowej z egzaminu gdy student nie posiada pozytywnej oceny z danych zajęć.
\\
\\
Zastosowanie walidacji w elektronicznym systemie wystawiania ocen udoskonali jego funkcjonalność,
a także usprawni działanie danej aplikacji. Nie dopuści również do wprowadzenia błędnych danych do systemu
oszczędzając tym samym czas użytkownika.




\chapter{Jak stworzyć pakiet do Meteora?}


\section{Co to jest pakiet?}

\section{Co zawieraja pakiety}

\section{Jak stworzyc własny pakiet}

\section{Jak opublikować swój pakiet}



\chapter{Jak testowane są pakiety Meteora?}

\section{Jak testować pakiety}

\section{Czemu testy sa takie ważne}

%\section{Stratosphere}

%\section{Testowanie pakietów}

\chapter{Aplikacja w Meteorze}

\chapter{Testowanie aplikacji w Meteorze}

% zakończenie
\summary

% załączniki (opcjonalnie):
%\appendix
%\chapter{Tytuł załącznika jeden}

%Treść załącznika jeden.

%\chapter{Tytuł załącznika dwa}

%Treść załącznika dwa.

\cite{MeteorDocs}
\cite{NodeDocs}
\cite{MongoDocs}
\cite{Mesosphere}
\cite{DiscoverMeteor2013}
\cite{ScalingMongoDB2011}
\cite{ScalingWithMongoDB}
\cite{Laika}
\cite{TinyTest}

% literatura (obowiązkowo):
%\bibliographystyle{unsrt}
%\bibliography{xml}

\begin{bibdiv}
\begin{biblist}

  \bibselect{mgr}

\end{biblist}
\end{bibdiv}

% spis tabel (jeżeli jest potrzebny):
%\listoftables

% spis rysunków (jeżeli jest potrzebny):
%\listoffigures

\oswiadczenie

\end{document}

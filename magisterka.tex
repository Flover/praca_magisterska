\documentclass[brudnopis]{xmgr}

\definecolor{sa}{cmyk}{0,1,0.13,0}
\definecolor{sb}{cmyk}{0.99,0,0.52,0}
\definecolor{sc}{cmyk}{0,0.75,1,0.24}
\definecolor{wa}{cmyk}{0,1,0.13,0}
\definecolor{wb}{cmyk}{0.99,0,0.52,0}
\definecolor{wc}{cmyk}{0,0.75,1,0.24}
\definecolor{wd}{cmyk}{0.98,0.13,0,0.43}
\definecolor{stressp}{cmyk}{0,1,0.13,0}
\definecolor{topicp}{cmyk}{0.98,0.13,0,0.43}

%\defaultfontfeatures{Scale=MatchLowercase}
%\setmainfont[Numbers=OldStyle,Ligatures=TeX]{Minion Pro}
%\setsansfont[Numbers=OldStyle,Ligatures=TeX]{Myriad Pro}
% for fontspec version < 2.0
\setmainfont[Numbers=OldStyle,Mapping=tex-text]{Minion Pro}
\setsansfont[Numbers=OldStyle,Mapping=tex-text]{Myriad Pro}
%\setmonofont[Scale=0.75]{Monaco}

% Opcjonalnie identyfikator dokumentu
% drukowany tylko z włączoną opcją 'brudnopis':
\wersja   {wersja wstępna [\ymdtoday]}

\author   {Mateusz Kwiatkowski}
\nralbumu {194\,925}
\email    {emflover@gmail.com}

\kierunek{\textbf{INFORMATYKA}}

\title    {Walidacja w elektronicznym systemie zarządzania osiągnięciami studenta}
\date     {2014}
\miejsce  {Gdańsk}

\opiekun  {dr Włodzimierz Bzyl}

% dodatkowe polecenia

\begin{document}

\begin{abstract}

\begin{description}
\item[sa] \textcolor{sa}{powinno zawierać omówienie głównych 
tez pracy magisterskiej, celów jakie autor sobie postawił} 
\item[sb] \textcolor{sb}{powinno zawierać informację czy udało 
  się je zrealizować}
\item[sc] \textcolor{sc}{należy także napisać jakimi metodami,
  technologiami się posłużono i~jakie to przyniosło efekty}
\end{description}

\indent \indent \textcolor{sa}{Pracę poświęcono zagadnieniu walidacji, kwestii ważnej i integralnie związanej z odpowiednim funkcjonowaniem sieci.
W szczególności zwrócono uwagę na aspekt prawidłowego zarządzania jej jakością, co obligatoryjnie wiąże się z problemem
odpowiedniego zabezpieczenia i odpowiedniego korzystania z niej.}
\\
\indent \textcolor{sa}{Praca prezentuje sposób tworzenia oraz funkcjonowanie pakietu walidującego do frameworka \textit{Meteor}. Pakiet ten
ma uniemożliwić użytkownikowi wprowadzenie błędnych danych do systemu, dzięki czemu podniesiony zostanie poziom zaufania
do korzystania z niego. Jednocześnie, aby przybliżyć i zademonstrować jego działanie, utworzono na potrzeby pracy aplikację elektroniczny indeks.
Wybór dokumentu nie był przypadkowy. Świadczy o tym jego wysoka ranga wśród uczelnianej dokumentacji urzędowej. Inny, równie istotny powód
wyboru stanowi fakt, iż korzystanie z sieci komputerowej w systemie edukacyjnym stało sie powszechne. Uczelnie wyższe wykorzystują sieć by
między innymi ułatwić kontakty na lini: wykładowca - student - administracja uczelni.  Temu ma służyć wprowadzenie w ostatnich latach przez
większość uczelni wyższych w Polsce, w tym Uniwersytet Gdański, elektronicznego systemu zarządzania osiągnieciami studentów
tzw. elektroniczny indeks.}
\\
\indent \textcolor{sc}{W pracy walidacji zostana poddane operacje, które można wykonać w aplikacji elektroniczny indeks. Do stworzenia jej użyto frameworku \textit{Meteor}, a dane wprowadzone do systemu, w celu ich przechowywania umieszczono w bazie danych - \textit{MongoDB}.}
\\
\indent \textcolor{sb}{Założono, że skutkiem tego informatycznego sofizmatu będzie uproszczenie, a nawet intuicyjność obsługi oprogramowania. Cele założone przez autora pracy zostały zrealizowane, czego dowodem jest udostepnienie do pobrania pakietu walidacji systemu.}

\end{abstract}
%\keywords{SGML,
% dokumenty strukturalne}

% tytuł i spis treści
\maketitle
%
% wstęp
\introduction

\begin{description}
\item[wa] \textcolor{wa}{jak nazwa wskazuje, ma wprowadzać 
  w~obszar problemowy pracy}
\item[wb] \textcolor{wb}{powinno przedstawiać ogólne 
  uwarunkowania problemu oraz opisać go w~kontekście}
\item[wc] \textcolor{wc}{powinno zawierać powód dlaczego 
  poruszyło się taki temat}
\item[wd] \textcolor{wd}{należy odnieść się do dorobku innych}
\end{description}

\indent \indent \textcolor{wa}{Najcenniejszym walorem komputera i internetu są przechowywane w nich dane - zarówno
ich ilość, jak i jakość. Ze względu na to, z dnia na dzień, rośnie liczba
użytkowników sieci. Jednocześnie zwiększa się liczebność i różnorodność usług
sieciowych.}
\\
\indent \textcolor{wa}{Komputer i internet zmienił i wciąż zmienia naszą codzienność. To prawda oczywista.
Usługi internetowe nie są już domeną urzędów, firm czy handlu. Chcemy za ich pomocą
robić zakupy, obsługiwać konto w banku, a także załatwiać wszelkie formalności w
urzędach. Jest to po prostu wymóg rozwoju cywilizacji, techniki oraz oszczędności
czasu.}
\\
\indent \textcolor{wa}{Coraz częściej systemy informatyczne wykorzystywane są w edukacji społeczeństwa.
Jeszcze do niedawna na wszystkich uczelniach wyższych stosowano klasyczne indeksy
papierowe, aby zarchiwizować osiągnięcia studentów podczas całego cyklu kształcenia.
Jednak w wyniku rozwoju technologii internetowych coraz częściej rezygnuje się
z klasycznych rozwiązań, zastępując je ich elektronicznymi odpowiednikami.}
\\
\indent \textcolor{wb}{Dziś wiele szkół i uczelni wprowadziło do obszaru swego funkcjonowania nowoczesny
system ewidencji osiągnięć ucznia czy studenta. W szkołach podstwowych, gimnazjach,
liceach, technikach czy zasadniczych szkołąch zawodowych jest nim tzw. dziennik
elektroniczny. W uczelniach wyższych  nazwano go elektronicznym indeksem. Zjawisko
to stanowi nie lada wyzwanie, ponieważ wiąże się z problemem niezawodnego świadczenia
usług w sieci komputerowej. Odbiorca, w tym przypadku uczeń lub student, musi mieć
pewność, że dane są stałe, prawdziwe, odpowiednio zabezpieczone przed ich utratą
czy nieuprawnionym dostepem. Należy nadmienić, że taki poziom zaufania i poczucia
bezpieczeństwa funkcjonowania systemu, powinna mieć również druga strona - nadawca,
ten który wprowadza owe dane. Jest o tym bardziej ważną kwestią, gdyż coraz częściej
mamy do czynienia ze zdarzeniami, wskazującymi na nieprawidłowe stosowanie sieci
komputerowej lub jej nadużycie.}
\\
\indent \textcolor{wa}{Rozwiązaniem, które zapewniłoby wzrost poziomu zaufania do korzystania
\\
z sieci,
w tym również z elektronicznego systemu zarządzania osiagnieciami ucznia lub studenta
jest, według autora niniejszej pracy, odpowiednie i odpowiedzialne zarządzanie jej
jakością, czemu służy walidacja systemu.}
\\
\indent \textcolor{wa}{Zjawisko to jest szeroko stosowane w technice i informatyce. Internetowy Słownik
Języka Polskiego wyjaśnia hasło ,,walidacja'' w następujacy sposób: ,,walidacja
(technika) - badanie odpowiedności, trafnośc lub dokładności czegoś''.\cite{ValidationSJP}}
\\
\indent \textcolor{wa}{Sam termin - ,,walidacja'' pochodzi od angielskiego słowa ,,validate'' i oznacza -
w kontekście informatycznym - sprawdzanie poprawności i zgodności z zadanymi
kryteriami. Jest on stosowany w odniesieniu do danych pochodzących od użytkownika,
jak również w stosunku do zmiennych, obiektów, typów i klas w różnych językach
programowania.\cite{ValidationTermin}}
\\
\indent \textcolor{wa}{Walidacja jest działaniem, mającym na celu potwierdzenie w sposób udokumentowany
i zgodny z założeniami, że procedury, procesy, urządzenia, materiały, czynności
i systemy, rzeczywiście prowadzą do zaplanowanych wyników. Znana jest także jako
kontrola jakości oprogramowania.\cite{Validation2}}
\\
\indent \textcolor{wa}{Wprowadzając dane do systemu, użytkownik może - świadomie lub nie - popełnić
pomyłkę. Jeżeli dane odebrane przez użytkownika poddamy przetworzeniu bez weryfikacji,
wówczas, w zależności od odporności aplikacji, możemy mieć do czynienia z różnymi
rodzajami błędów, od drukowania w przeglądarce klienta komunikatów diagnostycznych,
poprzez utratę spójności bazy danych, aż po ujawnienie niepowołanym użytkownikom
informacji poufnych. Z tego powodu nie wolno ignorować wagi problemu.}
\\
\indent \textcolor{wc}{Aplikacje pozbawione walidacji pozwalają użytkownikowi na wprowadzenie irracjonalnych
danych do systemu. Przykładem takiej aplikacji jest wspomiany prez autora pracy
elektroniczny indeks. Operacje, takie jak: wystawianie studentowi ocen z ćwiczeń
czy też oecny z egzaminu kończącej edukację z danego przedmiotu, powinny być
odpowiednio walidowane. Dzięki temu nie dojdzie do niepożądanych zjawisk typu:
\begin{itemize}
\item student nie uzyskał pozytywnej oceny z ćwiczeń, a otrzymuje ocenę z egzaminu
kończącego przedmiot,
\item student otrzymuje ocenę spoza skali oceniania systemu danej uczelni,
\item student uzyskuje ocenę od osoby nieuprawnionej do jej wystawienia.
\end{itemize}}

\textcolor{wc}{Dlatego też autor pracy chce zwrócić uwagę na rodzący się problem związany
z wprowadzeniem przez uczelnie elektronicznego indeksu oraz jego odpowiednim
funkcjonowaniem. Zaproponowanie zastosowania walidacji w elektronicznym systemie
wystawiania ocen usprawni działanie oraz udoskonali jego funkcjonalność.
Korzystając z aplikacji, w której zaimplementowana jest walidacja możemy mieć pewność,
że nie dojdzie do sytuacji, by użytkownik wprowadził błędne dane do systemu.
Należy również zwrócić uwagę na ekonomiczny aspekt walidacji. Mianowicie oszczędność
czasu użytkownika czy zwiększenie efektywności jego pracy.}
\\
\indent \textcolor{wc}{W celu ukazania i udowodnienia przydatności walidacji podczas korzystania
z elektronicznego systemu zarządzania osiągnięciami studenta, pokazano w pracy
działanie tego tego zjawiska w aplikacji stworzonej w frameworku \textit{Meteor} oraz
zaprezentowano ułożony pakiet oraz wyjaśniono, jak udostępnić go w prosty, jasny
i zrozumiały sposób.}
\\
\indent \textcolor{wd}{Tworzenie pakietu walidującego oraz aplikacji - elektroniczny indeks, która korzysta
ze stworzonego w ramach pracy pakietu, oparto na doświadczeniu innych badaczy,
zajmujących się oprogramowaniami komputerowymi, takich jak: Kelly Copley, Tom
Coleman czy Sacha Greif.} \textcolor{wa}{W pracy umieszczono ponadto uzasadnienie, dlaczego wybrane
technologie, takie jak - \textit{Meteor} oraz \textit{MongoDB} to najbardziej trafny wybór do generowania
pakietu walidacyjnego elektronicznego zarządzania osiągnięciami studenta.}
\\
\indent \textcolor{wa}{Autor niniejszej pracy miał kontakt z wieloma systemami zarządzania osiągnieciami
studentów, ale w każdym mozna było doprowadzać do anomalii. Zajęcie się rozwiązaniem
tego problemu jest, z punktu widzenia informatyka interesujące. Efektem pracy może być
nie tylko usprawnienie działania systemu, ale również poczucie, że praca z nim jest
prosta, przyjemna i wręcz intuicyjna.}





\chapter{Walidacja oprogramowania}

\begin{description}
\item[stressp] \textcolor{sa}{Najbardziej istotna informacja} 
\item[topicp] \textcolor{sb}{Początek nowego zdania odnoszący sie do istotnej informacji poprzedniego zdania}
\end{description}

\section{Wstęp do walidacji}
\cite{Validation}
\section{Typy walidacji}

\chapter{Aplikacja Elektroniczny indeks w Meteor}

\section{Cele i funkcjonalność aplikacji}
\section{Opis tworzenia aplikacji}
\cite{Introduction}
\cite{MeteorDocs}
\cite{DiscoverMeteor2013}
\cite{NodeDocs}
\cite{MongoDocs}
\cite{ScalingMongoDB2011}
\cite{ScalingWithMongoDB}
\section{Opis testowania aplikacji}
\cite{Laika}
\section{Opis własnych rozwiązań}


\chapter{Pakiet walidujący operacje elektronicznego indeksu}

\section{Funkcjonalność pakietu}
\section{Opis tworzenia pakietu}
\cite{Packages}
\cite{MeteorDocs}
\cite{DiscoverMeteor2013}
\section{Implementacja pakietu w aplikacji}
\section{Przetestowanie pakietu}
\cite{TinyTest}

% zakończenie
\summary

% załączniki (opcjonalnie):
\appendix
\chapter{Mesosphere}

%Treść załącznika jeden.

\chapter{Meteor}

%Treść załącznika dwa.

\chapter{Laika}

%Treść załącznika trzy.

\chapter{TinyTest}

%Treść załącznika cztery.



% literatura (obowiązkowo):
\bibliographystyle{unsrt}
\bibliography{mgr}

% spis tabel (jeżeli jest potrzebny):
\listoftables

% spis rysunków (jeżeli jest potrzebny):
\listoffigures

\oswiadczenie

\end{document}

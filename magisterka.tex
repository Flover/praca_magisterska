\documentclass[brudnopis]{xmgr}

%\defaultfontfeatures{Scale=MatchLowercase}
%\setmainfont[Numbers=OldStyle,Ligatures=TeX]{Minion Pro}
%\setsansfont[Numbers=OldStyle,Ligatures=TeX]{Myriad Pro}
% for fontspec version < 2.0
\setmainfont[Numbers=OldStyle,Mapping=tex-text]{Minion Pro}
\setsansfont[Numbers=OldStyle,Mapping=tex-text]{Myriad Pro}
%\setmonofont[Scale=0.75]{Monaco}

% Opcjonalnie identyfikator dokumentu 
% drukowany tylko z włączoną opcją 'brudnopis':
\wersja   {wersja wstępna [\ymdtoday]}

\author   {Mateusz Kwiatkowski}
\nralbumu {194\,925}
\email    {emflover@gmail.com}

\kierunek{\textbf{INFORMATYKA}}

\title    {Walidacja w MeteorJS na przykładzie elektronicznego indeksu}
\date     {2014}
\miejsce  {Gdańsk}

\opiekun  {dr Włodzimierz Bzyl}

% dodatkowe polecenia

\begin{document}

\begin{abstract}
%STRESZCZENIE
\end{abstract}
%\keywords{SGML, 
% dokumenty strukturalne}

% tytuł i spis treści
\maketitle
%
% wstęp
\introduction

Meteor jest dość nowym frameworkiem który bazuje na pakietach, więc jego podstawowa funkcjonalność nie jest kompletna.
Można bardzo mocno zwiększyć jego możliwości dodając pakiety, które otrzymujemy od twórców Meteora lub od społeczności.
Jedną z podstawowych funkcji jakiej meteor domyślnie nie posiada jest walidacja, ale powstał już pakiet, który doskonale radzi
sobie z podstawową walidacją.
\\
\\
Jeszcze do niedawna na wszystkich uczelniach stosowano klasyczne indeksy
papierowe, jednak w wyniku rozwoju technologii internetowych coraz częściej
rezygnuje się z klasycznych rozwiązań zastępując je ich elektronicznymi odpowiednikami.
Korzystając z elektronicznego indeksu jedną z ważniejszych funkcji jest walidacja. Aplikacja nie może dopuścić do
sytuacji gdzie nauczyciel wystawi studentowi ocenę spoza skali czy też wprowadzić niepełne dane, ale również
poprawnie interpetować czy dany użytkownik może wykonać w danej konkretną akcję. O ile do tej prostszej części
walidacji istnieje pakiet to do tej bardziej zaawansowanej już takiego nie znajdziemy więc będzie trzeba go stworzyć.
\\
\\
Na każdej uczelni znajduje się wielu studentów oraz wykładowców przez co trzeba zadbać. W momencie
gdy wiele osób jednocześnie chce przejrzeć indeks lub wystawić oceny nie można dopuścić do sytuacji gdy nasza
aplikacja nie będzie w stanie obsłużyć wszystkich osób w jednym czasie. Z pomocą przychodzi nam javascriptowy
framework \textbf{MeteorJS} który w połączeniu z bazą danych \textbf{MongoDB} zapewni nam wystarczającą skalowalność
aplikacji oraz dostarczy nam sporą ilość gotowych pakietów, które uproszczą stworzenie nowego produktu.




\chapter{Walidacja}


    
\section{Walidacja w MeteorJS}

\section{MeteorJS i Mesosphere}


      
\chapter{Pakiet do walidacji}

\section{Meteorite i Atmospherejs}
    
\section{Co zawierają pakiety}

\section{Stratosphere}
   
\section{Testowanie pakietów}

\chapter{Aplikacja w MeteorJS}

\chapter{Testowanie aplikacji}

\chapter{Porównanie z obecnie używanym elektronicznym indeksem}

% zakończenie 
\summary

% załączniki (opcjonalnie):
%\appendix
%\chapter{Tytuł załącznika jeden}

%Treść załącznika jeden.

%\chapter{Tytuł załącznika dwa}

%Treść załącznika dwa.

\cite {Atmosphere}
\cite {Meteor}
\cite{Node}
\cite{Mongo}
\cite{Mesosphere}
\cite{DiscoverMeteor}
\cite{ScalingMongoDB}
\cite{ScalingWithMongoDB}
\cite{TDDwM}

% literatura (obowiązkowo):
\bibliographystyle{unsrt}
\bibliography{xml}

% spis tabel (jeżeli jest potrzebny):
%\listoftables

% spis rysunków (jeżeli jest potrzebny):
%\listoffigures

\oswiadczenie

\end{document}

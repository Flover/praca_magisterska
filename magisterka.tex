\documentclass[brudnopis]{xmgr}

\definecolor{sa}{cmyk}{0,1,0.13,0}
\definecolor{sb}{cmyk}{0.99,0,0.52,0}
\definecolor{sc}{cmyk}{0,0.75,1,0.24}
\definecolor{wa}{cmyk}{0,1,0.13,0}
\definecolor{wb}{cmyk}{0.99,0,0.52,0}
\definecolor{wc}{cmyk}{0,0.75,1,0.24}
\definecolor{wd}{cmyk}{0.98,0.13,0,0.43}
\definecolor{stressp}{cmyk}{0,1,0.13,0}
\definecolor{topicp}{cmyk}{0.98,0.13,0,0.43}

%\defaultfontfeatures{Scale=MatchLowercase}
%\setmainfont[Numbers=OldStyle,Ligatures=TeX]{Minion Pro}
%\setsansfont[Numbers=OldStyle,Ligatures=TeX]{Myriad Pro}
% for fontspec version < 2.0
\setmainfont[Numbers=OldStyle,Mapping=tex-text]{Minion Pro}
\setsansfont[Numbers=OldStyle,Mapping=tex-text]{Myriad Pro}
%\setmonofont[Scale=0.75]{Monaco}

% Opcjonalnie identyfikator dokumentu
% drukowany tylko z włączoną opcją 'brudnopis':
\wersja   {wersja wstępna [\ymdtoday]}

\author   {Mateusz Kwiatkowski}
\nralbumu {194\,925}
\email    {emflover@gmail.com}

\kierunek{\textbf{INFORMATYKA}}

\title    {Walidacja w elektronicznym systemie zarządzania osiągnięciami studenta}
\date     {2014}
\miejsce  {Gdańsk}

\opiekun  {dr Włodzimierz Bzyl}

% dodatkowe polecenia

\begin{document}

\begin{abstract}

\begin{description}
\item[sa] \textcolor{sa}{powinno zawierać omówienie głównych 
tez pracy magisterskiej, celów jakie autor sobie postawił} 
\item[sb] \textcolor{sb}{powinno zawierać informację czy udało 
  się je zrealizować}
\item[sc] \textcolor{sc}{należy także napisać jakimi metodami,
  technologiami się posłużono i~jakie to przyniosło efekty}
\end{description}

\textcolor{sa}{Pracę poświęcono zagadnieniu walidacji, kwestii ważnej i integralnie związanej z odpowiednim funkcjonowaniem sieci.
W szczególności zwrócono uwagę na aspekt prawidłowego zarządzania jej jakością, co obligatoryjnie wiąże się z problemem
odpowiedniego zabezpieczenia i odpowiedniego korzystania z niej.}
\\
\textcolor{sa}{Praca prezentuje sposób tworzenia oraz funkcjonowanie pakietu walidującego do frameworka \textit{Meteor}. Pakiet ten
ma uniemożliwić użytkownikowi wprowadzenie błędnych danych do systemu, dzięki czemu podniesiony zostanie poziom zaufania
do korzystania z niego. Jednocześnie, aby przybliżyć i zademonstrować jego działanie, utworzono na potrzeby pracy aplikację elektroniczny indeks.
Wybór dokumentu nie był przypadkowy. Świadczy o tym jego wysoka ranga wśród uczelnianej dokumentacji urzędowej. Inny, równie istotny powód
wyboru stanowi fakt, iż korzystanie z sieci komputerowej w systemie edukacyjnym stało sie powszechne. Uczelnie wyższe wykorzystują sieć by
między innymi ułatwić kontakty na lini: wykładowca - student - administracja uczelni.  Temu ma służyć wprowadzenie w ostatnich latach przez
większość uczelni wyższych w Polsce, w tym Uniwersytet Gdański, elektronicznego systemu zarządzania osiągnieciami studentów
tzw. elektroniczny indeks.}
\\
\textcolor{sc}{W pracy walidacji zostana poddane operacje, które można wykonać w aplikacji elektroniczny indeks. Do stworzenia jej użyto frameworku \textit{Meteor}, a dane wprowadzone do systemu, w celu ich przechowywania umieszczono w bazie danych - \textit{MongoDB}.}
\\
\textcolor{sb}{Założono, że skutkiem tego informatycznego sofizmatu będzie uproszczenie, a nawet intuicyjność obsługi oprogramowania. Cele założone przez autora pracy zostały zrealizowane, czego dowodem jest udostepnienie do pobrania pakietu walidacji systemu.}

\end{abstract}
%\keywords{SGML,
% dokumenty strukturalne}

% tytuł i spis treści
\maketitle
%
% wstęp
\introduction

\begin{description}
\item[wa] \textcolor{wa}{jak nazwa wskazuje, ma wprowadzać 
  w~obszar problemowy pracy}
\item[wb] \textcolor{wb}{powinno przedstawiać ogólne 
  uwarunkowania problemu oraz opisać go w~kontekście}
\item[wc] \textcolor{wc}{powinno zawierać powód dlaczego 
  poruszyło się taki temat}
\item[wd] \textcolor{wd}{należy odnieść się do dorobku innych}
\end{description}

\textcolor{wa}{Walidacja jest działaniem mającym na celu potwierdzenie w sposób udokumentowany i zgodny
z założeniami, że procedury, procesy, urządzenia, materiały, czynności i systemy rzeczywiście
prowadzą do zaplanowanych wyników. Znana jest także jako kontrola jakości oprogramowania.
Wykorzystuje się ją w naukach technicznych oraz informatyce.}
\\

\textcolor{wa}{Aplikacje pozbawione walidacji pozwalają użytkownikowi na wprowadzenie
irracjonalnych danych do systemu.} \textcolor{wb}{Przykładem takiej aplikacji jest elektroniczny indeks.
Operacje takie jak wystawianie studentowi ocen z ćwiczeń czy też wystawienie
oceny końcowej z egzaminu gdy student nie posiada pozytywnej oceny
\\
z danych zajęć powinny być odpowiednio
walidowane i nie dopuszczać do sytuacji gdy student otrzymuje ocenę z poza skali lub od osoby do tego nieupoważnionej.}
\textcolor{wb}{Jeszcze do niedawna na wszystkich uczelniach stosowano klasyczne indeksy papierowe,
jednak w wyniku rozwoju technologii internetowych coraz częściej rezygnuje się z klasycznych
rozwiązań zastępując je ich elektronicznymi odpowiednikami.}
\\

\textcolor{wc}{Zastosowanie walidacji w elektronicznym
systemie wystawiania ocen udoskonali jego funkcjonalność, a także usprawni działanie takiej aplikacji.
Korzystając 
\\
z aplikacji w której zaimplementowana jest walidacja nie dojdzie do sytuacji gdzie użytkownik wprowadzi błędne dane do systemu oszczędzając tym samym jego czas, a także zwiększy jego efektywność. 
Miałem kontakt z wieloma systemami zarządzania osiągnieciami studentów, ale w każdym można było doprowadzać
do anomalii, a samo działanie takiej aplikacji również pozostawiało wiele do życzenia, dlatego postanowiłem zająć się tym tematem,
aby usprawnić działanie takiego systemu oraz żeby praca na nim była przyjemna, prosta i intuicyjna. }
\\

\textcolor{wa}{Postaram się udowodnić jak bardzo przydatna jest walidacja pokazując jej działanie w aplikacji
stworzonej w frameworku \textit{Meteor}. Pokażę również na czym polega stworzenie pakietu i udostepnienie
go w prosty sposób.} \textcolor{wd}{Opierając się na doświadczeniach innych badaczy takich jak Kelly Copley \cite{Mesosphere}
czy Tom Coleman i Sacha Greif \cite{DiscoverMeteor2013} napiszę pakiet walidujący oraz aplikację Elektroniczny Indeks,
która będzie korzystać ze stworzonego w ramach pracy pakietu.} \textcolor{wa}{Napiszę także dlaczego uważam \textit{Meteor} oraz \textit{MongoDB} jako najlepszy wybór.}




\chapter{Walidacja oprogramowania}

\begin{description}
\item[stressp] \textcolor{sa}{Najbardziej istotna informacja} 
\item[topicp] \textcolor{sb}{Początek nowego zdania odnoszący sie do istotnej informacji poprzedniego zdania}
\end{description}

\section{Wstęp do walidacji}
\cite{Validation}
\section{Typy walidacji}

\chapter{Aplikacja Elektroniczny indeks w Meteor}

\section{Cele i funkcjonalność aplikacji}
\section{Opis tworzenia aplikacji}
\cite{Introduction}
\cite{MeteorDocs}
\cite{DiscoverMeteor2013}
\cite{NodeDocs}
\cite{MongoDocs}
\cite{ScalingMongoDB2011}
\cite{ScalingWithMongoDB}
\section{Opis testowania aplikacji}
\cite{Laika}
\section{Opis własnych rozwiązań}


\chapter{Pakiet walidujący operacje elektronicznego indeksu}

\section{Funkcjonalność pakietu}
\section{Opis tworzenia pakietu}
\cite{Packages}
\cite{MeteorDocs}
\cite{DiscoverMeteor2013}
\section{Implementacja pakietu w aplikacji}
\section{Przetestowanie pakietu}
\cite{TinyTest}

% zakończenie
\summary

% załączniki (opcjonalnie):
\appendix
\chapter{Mesosphere}

%Treść załącznika jeden.

\chapter{Meteor}

%Treść załącznika dwa.

\chapter{Laika}

%Treść załącznika trzy.

\chapter{TinyTest}

%Treść załącznika cztery.



% literatura (obowiązkowo):
\bibliographystyle{unsrt}
\bibliography{mgr}

% spis tabel (jeżeli jest potrzebny):
\listoftables

% spis rysunków (jeżeli jest potrzebny):
\listoffigures

\oswiadczenie

\end{document}

\documentclass[brudnopis]{xmgr}

\definecolor{sa}{cmyk}{0,1,0.13,0}
\definecolor{sb}{cmyk}{0.99,0,0.52,0}
\definecolor{sc}{cmyk}{0,0.75,1,0.24}
\definecolor{wa}{cmyk}{0,1,0.13,0}
\definecolor{wb}{cmyk}{0.99,0,0.52,0}
\definecolor{wc}{cmyk}{0,0.75,1,0.24}
\definecolor{wd}{cmyk}{0.98,0.13,0,0.43}
\definecolor{stressp}{cmyk}{0,1,0.13,0}
\definecolor{topicp}{cmyk}{0.98,0.13,0,0.43}

%\defaultfontfeatures{Scale=MatchLowercase}
%\setmainfont[Numbers=OldStyle,Ligatures=TeX]{Minion Pro}
%\setsansfont[Numbers=OldStyle,Ligatures=TeX]{Myriad Pro}
% for fontspec version < 2.0
\setmainfont[Numbers=OldStyle,Mapping=tex-text]{Minion Pro}
\setsansfont[Numbers=OldStyle,Mapping=tex-text]{Myriad Pro}
%\setmonofont[Scale=0.75]{Monaco}

% Opcjonalnie identyfikator dokumentu
% drukowany tylko z włączoną opcją 'brudnopis':
\wersja   {wersja wstępna [\ymdtoday]}

\author   {Mateusz Kwiatkowski}
\nralbumu {194\,925}
\email    {emflover@gmail.com}

\kierunek{\textbf{INFORMATYKA}}

\title    {Walidacja w aplikacjach Meteor}
\date     {2014}
\miejsce  {Gdańsk}

\opiekun  {dr Włodzimierz Bzyl}

% dodatkowe polecenia

\begin{document}

\begin{abstract}

\begin{description}
\item[sa] \textcolor{sa}{powinno zawierać omówienie głównych 
tez pracy magisterskiej, celów jakie autor sobie postawił} 
\item[sb] \textcolor{sb}{powinno zawierać informację czy udało 
  się je zrealizować}
\item[sc] \textcolor{sc}{należy także napisać jakimi metodami,
  technologiami się posłużono i~jakie to przyniosło efekty}
\end{description}

\textcolor{sa}{W pracy przedstawiony zostanie pakiet walidujący do frameworka \textit{Meteor}. Do zobrazowania
jego działania zostanie również stworzona aplikacja Elektroniczny Indeks. Pakiet do walidacji ma na celu zapobiegać
wprowadzaniu przez użytkownika błędnych danych do systemu. Chcąc to osiągnąć będzie
on walidować operację wstawiania ocen w dostępnej skali, uniemożliwić zmianę ocen po
upływie ostatecznego terminu czy też wystawienie oceny pozytywnej przy braku zaliczenia zajęć
z danego przedmiotu, a także obliczanie średniej ocen jedynie z egzaminów.} \textcolor{sc}{Do stworzenia aplikacji zostanie użyty
framework \textit{Meteor}, a dane wprowadzane do systemu będa przechowywane w bazie danych \textit{MongoDB}. Opiszę również
dlaczego skorzystałem własnie z tych technologii.}
\textcolor{sb}{Wszystkie te cele zostały zrealizowane, a pakiet do walidacji został udostępniony do pobrania.}

\end{abstract}
%\keywords{SGML,
% dokumenty strukturalne}

% tytuł i spis treści
\maketitle
%
% wstęp
\introduction

\begin{description}
\item[wa] \textcolor{wa}{jak nazwa wskazuje, ma wprowadzać 
  w~obszar problemowy pracy}
\item[wb] \textcolor{wb}{powinno przedstawiać ogólne 
  uwarunkowania problemu oraz opisać go w~kontekście}
\item[wc] \textcolor{wc}{powinno zawierać powód dlaczego 
  poruszyło się taki temat}
\item[wd] \textcolor{wd}{należy odnieść się do dorobku innych}
\end{description}

Walidacja jest działaniem mającym na celu potwierdzenie w sposób udokumentowany i zgodny
z założeniami, że procedury, procesy, urządzenia, materiały, czynności i systemy rzeczywiście
prowadzą do zaplanowanych wyników. Wykorzystuje się ją w naukach technicznych oraz informatyce.
Przykładem wykorzystania tej techniki jest elektroniczny indeks.
\\
\\
Jeszcze do niedawna na wszystkich uczelniach stosowano klasyczne indeksy papierowe,
jednak w wyniku rozwoju technologii internetowych coraz częściej rezygnuje się z klasycznych
rozwiązań zastępując je ich elektronicznymi odpowiednikami. Elektroniczne indeksy pozbawione
walidacji pozwalają użytkownikowi na wprowadzenie irracjonalnych danych do systemu.
Przykładem takiego działania jest wystawienie studentowi oceny spoza skali czy też wystawienie
oceny końcowej z egzaminu gdy student nie posiada pozytywnej oceny z danych zajęć.
\\
\\
Zastosowanie walidacji w elektronicznym systemie wystawiania ocen udoskonali jego funkcjonalność,
a także usprawni działanie danej aplikacji. Nie dopuści również do wprowadzenia błędnych danych do systemu
oszczędzając tym samym czas użytkownika.




\chapter{Walidacja oprogramowania}

\section{Wstęp do walidacji}
\section{Typy walidacji}

\chapter{Aplikacja Elektroniczny indeks w Meteor}

\section{Cele i funkcjonalność aplikacji}
\section{Opis tworzenia aplikacji}
\section{Opis testowania aplikacji}
\section{Opis własnych rozwiązań}


\chapter{Pakiet walidujący operacje elektronicznego indeksu}

\section{Funkcjonalność pakietu}
\section{Opis tworzenia pakietu}
\section{Implementacja pakietu w aplikacji}
\section{Przetestowanie pakietu}

% zakończenie
\summary

% załączniki (opcjonalnie):
\appendix
\chapter{Mesosphere}

%Treść załącznika jeden.

\chapter{Meteor}

%Treść załącznika dwa.

\chapter{Laika}

%Treść załącznika trzy.

\chapter{TinyTest}

%Treść załącznika cztery.

\cite{MeteorDocs}
\cite{NodeDocs}
\cite{MongoDocs}
\cite{Mesosphere}
\cite{DiscoverMeteor2013}
\cite{ScalingMongoDB2011}
\cite{ScalingWithMongoDB}
\cite{Laika}
\cite{TinyTest}
\cite{Introduction}

% literatura (obowiązkowo):
\bibliographystyle{unsrt}
\bibliography{mgr}

% spis tabel (jeżeli jest potrzebny):
\listoftables

% spis rysunków (jeżeli jest potrzebny):
\listoffigures

\oswiadczenie

\end{document}

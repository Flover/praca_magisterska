\documentclass[brudnopis]{xmgr}
\usepackage{amsrefs}

%\defaultfontfeatures{Scale=MatchLowercase}
%\setmainfont[Numbers=OldStyle,Ligatures=TeX]{Minion Pro}
%\setsansfont[Numbers=OldStyle,Ligatures=TeX]{Myriad Pro}
% for fontspec version < 2.0
\setmainfont[Numbers=OldStyle,Mapping=tex-text]{Minion Pro}
\setsansfont[Numbers=OldStyle,Mapping=tex-text]{Myriad Pro}
%\setmonofont[Scale=0.75]{Monaco}

% Opcjonalnie identyfikator dokumentu 
% drukowany tylko z włączoną opcją 'brudnopis':
\wersja   {wersja wstępna [\ymdtoday]}

\author   {Mateusz Kwiatkowski}
\nralbumu {194\,925}
\email    {emflover@gmail.com}

\kierunek{\textbf{INFORMATYKA}}

\title    {Walidacja w aplikacjach Meteor}
\date     {2014}
\miejsce  {Gdańsk}

\opiekun  {dr Włodzimierz Bzyl}

% dodatkowe polecenia

\begin{document}

\begin{abstract}
%STRESZCZENIE
\end{abstract}
%\keywords{SGML, 
% dokumenty strukturalne}

% tytuł i spis treści
\maketitle
%
% wstęp
\introduction

\textit{Meteor} jest frameworkiem który bazuje na pakietach. Jego pierwsza odsłona miała miejsce w grudniu 2011 roku pod nazwą \textit{Skybreak}.
Z kolei w styczniu 2012 roku twórcy postanowili zmienić nazwę na \textit{Meteor}. 
\\
\\
Obecnie \textit{Meteor} jest dostępny w wersji beta, dlatego jego
podstawowa funkcjonalność nie jest kompletna. Można zwiększyć jego możliwości dodając pakiety, które otrzymujemy
od twórców Meteora oraz od społeczności.
Jedną z podstawowych funkcji jakiej \textit{Meteor} domyślnie nie posiada jest walidacja, ale powstał już pakiet, który doskonale radzi
sobie z podstawową walidacją co ma zastosowanie między innymi w elektronicznym indeksie.
\\
\\
Jeszcze do niedawna na wszystkich uczelniach stosowano klasyczne indeksy
papierowe, jednak w wyniku rozwoju technologii internetowych coraz częściej
rezygnuje się z klasycznych rozwiązań zastępując je ich elektronicznymi odpowiednikami.
Korzystając z elektronicznego indeksu jedną z ważniejszych funkcji jest właśnie walidacja. Aplikacja nie może dopuścić do
sytuacji gdzie nauczyciel wystawi studentowi ocenę spoza skali czy też wprowadzi niepełne dane, ale również musi
poprawnie interpetować czy dany użytkownik może wykonać w danej chwili konkretną akcję. O ile do tej pierwszej, a zarazem prostszej części
walidacji istnieje pakiet to do tej drugiej, bardziej zaawansowanej już takiego nie znajdziemy więc będzie trzeba go stworzyć ułatwiając tym samym
użytkowanie elektronicznego indeksu na uczelni.
\\
\\
Na każdej uczelni znajduje się wielu studentów oraz wykładowców przez co nie można dopuścić do sytuacji gdy nasza
aplikacja nie będzie w stanie obsłużyć wszystkich osób w jednym czasie. Z pomocą przychodzi nam javascriptowy
framework \textit{MeteorJS} który w połączeniu z bazą danych \textit{MongoDB} zapewni nam wystarczającą skalowalność
aplikacji oraz dostarczy nam sporą ilość gotowych pakietów, które uproszczą stworzenie nowego produktu.




\chapter{Jak stworzyć pakiet do Meteora?}


\section{Co to jest pakiet?}

\section{Co zawieraja pakiety}

\section{Jak stworzyc własny pakiet}

\section{Jak opublikować swój pakiet}


      
\chapter{Jak testowane są pakiety Meteora?}

\section{Czemu testy sa takie ważne}
    
\section{Jak testować pakiety}

%\section{Stratosphere}
   
%\section{Testowanie pakietów}

\chapter{Aplikacja w Meteorze}

\chapter{Testowanie aplikacji w Meteorze}

\chapter{Porównanie z obecnie używanym elektronicznym indeksem}

% zakończenie 
\summary

% załączniki (opcjonalnie):
%\appendix
%\chapter{Tytuł załącznika jeden}

%Treść załącznika jeden.

%\chapter{Tytuł załącznika dwa}

%Treść załącznika dwa.

\cite{MeteorDocs}
\cite{NodeDocs}
\cite{MongoDocs}
\cite{Mesosphere}
\cite{DiscoverMeteor2013}
\cite{ScalingMongoDB2011}
\cite{ScalingWithMongoDB}
\cite{Laika}
\cite{TinyTest}

% literatura (obowiązkowo):
%\bibliographystyle{unsrt}
%\bibliography{xml}

\begin{bibdiv}
\begin{biblist}

  \bibselect{mgr}

\end{biblist}
\end{bibdiv}

% spis tabel (jeżeli jest potrzebny):
%\listoftables

% spis rysunków (jeżeli jest potrzebny):
%\listoffigures

\oswiadczenie

\end{document}
